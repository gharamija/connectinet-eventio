\chapter{Implementacija i korisničko sučelje}
		
		
		\section{Korištene tehnologije i alati}
		
			Za međusobnu komunikaciju unutar tima koristili smo aplikacije WhatsApp i Discord. \textbf{\textit{WhatsApp}} je popularna platforma koja omogućava korisnicima slanje poruka i medijskih datoteka te obavljanje glasovnih/video poziva. \textbf{\textit{Discord}} je aplikacija za stvaranje servera na kojima članovi mogu komunicirati putem poruka i poziva. 
			
			Za modeliranje UML dijagrama koristili smo \textbf{\textit{Astah UML}}, alat koji omogućava korisnicima da kreiraju različite vrste UML dijagrama. Za pripremu i uređivanje dokumentacije koristili smo \textbf{\textit{TeXstudio}}, specifično prilagođeno okruženje za učinkovit rad s LaTeX sustavom.
			
			Po pitanju baze podataka i svega vezanog uz nju koristili smo nekoliko alata. Alatom \textbf{\textit{ERDplus}}, popularnim online alatom za izradu ER dijagrama, modelirali smo ER dijagram. Sama baza podataka temeljena je na \textbf{\textit{PostgreSQL}}-u, poznatom sustavu za upravljanje bazama podataka. \textbf{\textit{Firebase}} je platforma koja nam služi za pohranu slika koje organizatori postavljaju uz svoje događaje. Alat \textbf{\textit{Liquibase}} koristili smo za upravljanje promjenama u shemi baze podataka.
			
			Za ispitivanje komponenti iskoristili smo \textbf{\textit{JUnit}}, okvir za testiranje osmišljen za podršku automatskom testiranju Java aplikacija, i \textbf{\textit{Mockito}}, koji nam je poslužio za stvaranje lažnih, mock objekata. Za ispitivanje sustava upotrijebili smo \textbf{\textit{Selenium IDE}}, alat koji omogućuje snimanje, uređivanje i reprodukciju testova jednostavnim klikanjem i snimanjem korisničkih radnji na web stranici.
			
			Za upravljanje izvornim kodom primijenili smo sustav \textbf{\textit{Git}}, dok se sam kod nalazi na \textbf{\textit{GitHub}} platformi unutar udaljenog repozitorija. GitHub je jedna od najpoznatijih web platformi koja pruži usluge za upravljanje projektima temeljenima na Git sustavu.  
			
			Naposljetku, za izradu aplikacije upotrijebili smo mnogo različitih alata. Za frontend smo koristili \textbf{\textit{React.js}} uz \textbf{\textit{JavaScript}}, \textbf{\textit{Vite}} te \textbf{\textit{Material UI}}. React.js je popularna JavaScript biblioteka koja se koristi za izgradnju korisničkih sučelja u web aplikacijama. Vite je alat čija je osnovna svrha pojednostaviti i ubrzati razvoj web aplikacija. Material UI je poslužio za dizajniranje osnovnih komponenti frontenda. Za backend smo koristili \textbf{\textit{Javu}} i \textbf{\textit{Spring Boot}}, okvir za izgradnju Java aplikacija temeljenih na Springu, a koji je posebno koristan jer omogućuje programerima da se fokusiraju na poslovnu logiku aplikacije umjesto na konfiguraciju i upravljanje okolinom
			
			Deployment frontenda, backenda te baze podataka postigli smo uz pomoć platforme \textbf{\textit{Render}} koja služi za deployment i hosting web aplikacija. 
			
			\par\noindent\rule{\textwidth}{0.4pt}
			
				\begin{itemize}
					\item \footnotesize Whatsapp		\url{https://www.whatsapp.com/}
					\item \footnotesize Discord			\url{https://discord.com/}
					\item \footnotesize Astah UML		\url{https://astah.net/products/astah-uml/}
					\item \footnotesize TeXstudio		\url{https://www.texstudio.org/}
					\item \footnotesize ERDplus		\url{https://erdplus.com/}
					\item \footnotesize PostgreSQL		\url{https://www.postgresql.org/}
					\item \footnotesize Firebase		\url{https://firebase.google.com/}
					\item \footnotesize Liquibase		\url{https://www.liquibase.org/}
					\item \footnotesize JUnit			\url{https://junit.org/junit5/}
					\item \footnotesize Mockito		\url{https://site.mockito.org/}
					\item \footnotesize Selenium IDE		\url{https://www.selenium.dev/selenium-ide/}
					\item \footnotesize Git				\url{https://git-scm.com/}
					\item \footnotesize GitHub		\url{https://github.com/}
					\item \footnotesize React.js			\url{https://reactjs.org/}
					\item \footnotesize Vite			\url{https://vitejs.dev/}
					\item \footnotesize JavaScript		\url{https://www.javascript.com/}
					\item \footnotesize Material UI			\url{https://mui.com/}
					\item \footnotesize Java				\url{https://www.java.com/en/}
					\item \footnotesize SpringBoot			\url{https://spring.io/projects/spring-boot/}
					\item \footnotesize Render			 \url{https://render.com/}

				\end{itemize}

			
			
			
			\eject 
		
	
		\section{Ispitivanje programskog rješenja}
			
			\textbf{\textit{dio 2. revizije}}\\
			
			 \textit{U ovom poglavlju je potrebno opisati provedbu ispitivanja implementiranih funkcionalnosti na razini komponenti i na razini cijelog sustava s prikazom odabranih ispitnih slučajeva. Studenti trebaju ispitati temeljnu funkcionalnost i rubne uvjete.}
	
			
			\subsection{Ispitivanje komponenti}
			\textit{Potrebno je provesti ispitivanje jedinica (engl. unit testing) nad razredima koji implementiraju temeljne funkcionalnosti. Razraditi \textbf{minimalno 6 ispitnih slučajeva} u kojima će se ispitati redovni slučajevi, rubni uvjeti te izazivanje pogreške (engl. exception throwing). Poželjno je stvoriti i ispitni slučaj koji koristi funkcionalnosti koje nisu implementirane. Potrebno je priložiti izvorni kôd svih ispitnih slučajeva te prikaz rezultata izvođenja ispita u razvojnom okruženju (prolaz/pad ispita). }
			
			
			
			\subsection{Ispitivanje sustava}
			
			 \textit{Potrebno je provesti i opisati ispitivanje sustava koristeći radni okvir Selenium\footnote{\url{https://www.seleniumhq.org/}}. Razraditi \textbf{minimalno 4 ispitna slučaja} u kojima će se ispitati redovni slučajevi, rubni uvjeti te poziv funkcionalnosti koja nije implementirana/izaziva pogrešku kako bi se vidjelo na koji način sustav reagira kada nešto nije u potpunosti ostvareno. Ispitni slučaj se treba sastojati od ulaza (npr. korisničko ime i lozinka), očekivanog izlaza ili rezultata, koraka ispitivanja i dobivenog izlaza ili rezultata.\\ }
			 
			 \textit{Izradu ispitnih slučajeva pomoću radnog okvira Selenium moguće je provesti pomoću jednog od sljedeća dva alata:}
			 \begin{itemize}
			 	\item \textit{dodatak za preglednik \textbf{Selenium IDE} - snimanje korisnikovih akcija radi automatskog ponavljanja ispita	}
			 	\item \textit{\textbf{Selenium WebDriver} - podrška za pisanje ispita u jezicima Java, C\#, PHP koristeći posebno programsko sučelje.}
			 \end{itemize}
		 	\textit{Detalji o korištenju alata Selenium bit će prikazani na posebnom predavanju tijekom semestra.}
			
			\eject 
		
		
		\section{Dijagram razmještaja}
			
			\textbf{\textit{dio 2. revizije}}
			
			 \textit{Potrebno je umetnuti \textbf{specifikacijski} dijagram razmještaja i opisati ga. Moguće je umjesto specifikacijskog dijagrama razmještaja umetnuti dijagram razmještaja instanci, pod uvjetom da taj dijagram bolje opisuje neki važniji dio sustava.}
			
			\eject 
		
		\section{Upute za puštanje u pogon}
		
			\textbf{\textit{dio 2. revizije}}\\
		
			 \textit{U ovom poglavlju potrebno je dati upute za puštanje u pogon (engl. deployment) ostvarene aplikacije. Na primjer, za web aplikacije, opisati postupak kojim se od izvornog kôda dolazi do potpuno postavljene baze podataka i poslužitelja koji odgovara na upite korisnika. Za mobilnu aplikaciju, postupak kojim se aplikacija izgradi, te postavi na neku od trgovina. Za stolnu (engl. desktop) aplikaciju, postupak kojim se aplikacija instalira na računalo. Ukoliko mobilne i stolne aplikacije komuniciraju s poslužiteljem i/ili bazom podataka, opisati i postupak njihovog postavljanja. Pri izradi uputa preporučuje se \textbf{naglasiti korake instalacije uporabom natuknica} te koristiti što je više moguće \textbf{slike ekrana} (engl. screenshots) kako bi upute bile jasne i jednostavne za slijediti.}
			
			
			 \textit{Dovršenu aplikaciju potrebno je pokrenuti na javno dostupnom poslužitelju. Studentima se preporuča korištenje neke od sljedećih besplatnih usluga: \href{https://aws.amazon.com/}{Amazon AWS}, \href{https://azure.microsoft.com/en-us/}{Microsoft Azure} ili \href{https://www.heroku.com/}{Heroku}. Mobilne aplikacije trebaju biti objavljene na F-Droid, Google Play ili Amazon App trgovini.}
			
			
			\eject 