\chapter{Zaključak i budući rad}
		
		Cilj našeg projekta bio je razviti programsku podršku za web aplikaciju "Eventio", inovativnu platformu za promociju zabavnih događanja u gradu, koja bi korisnicima omogućila jednostavno stvaranje i sudjelovanje na događanjima. Ova inovativna web aplikacija je koncipirana kako bi korisnicima pružila lakši pristup društvenim događanjima koja odgovaraju njihovim interesima, istovremeno olakšavajući organizaciju i promociju tih događanja.
		
		Formiran je tim od sedam članova koji su u razdoblju od 17 tjedana surađivali na ovom projektu. Za svakodnevnu komunikaciju koristili smo WhatsApp, Discord za održavanje remote sastanaka te česte timske sastanke uživo kako bismo zajedno raspravljali o tijeku projekta, izazovima, i mogućim unapređenjima. Razvili smo internu raspodjelu zadataka koji su bili podijeljeni članovima tima što je osiguralo konstantan rad na projektu. Tome je pomogla i interna struktura tima podijeljena u podtimove s većim fokusom na drukčijim dijelovima projekta, uključujući backend, frontend i dokumentaciju. Iskusniji članovi su pružali mentorstvo i podršku onima s manje iskustva, olakšavajući im savladavanje novih tehnologija i koncepta.
		
		Prva faza projekta, počevši s okupljanjem tima, bila je ključna za razvoj međusobnih odnosa, što je rezultiralo učinkovitijom suradnjom u kasnijim fazama. Osim toga, faza je obuhvatila rad na dokumentaciji, uključujući izradu obrazaca uporabe, sekvencijske dijagrame, model baze podataka, dijagrame razreda. Izrada navedenih dijagrama, ali i vizualnih rješenja programskog zadatka bile su od ključne važnosti za usmjeravanje razvoja te su uvelike olakšale implementaciju rješenja.
		
		Druga faza projekta bila je usmjerena na konkretan razvoj web aplikacije. Osim samog programiranja, ova faza uključivala je detaljno dokumentiranje projekta, uključujući UML dijagrame (dijagram stanja, aktivnosti, komponenti i razmještaja) te izradu popratne dokumentacije koja je obuhvatila opis testiranja, deployment procese i druge važne aspekte implementacije.
		
		Gledajući prema budućnosti, projekt pruža niz mogućnosti za daljnje unapređenje. Jedna od mogućnosti je razvoj mobilne aplikacije kako bi se omogućilo korisnicima pristup događanjima putem mobilnih uređaja. Također, smatramo da bi uvođenje sustava za chat na događanjima dodatno obogatilo korisničko iskustvo omogućavajući sudionicima međusobnu komunikaciju. Osim toga, daljnje širenje na područja izvan Zagreba, primjerice na razini Republike Hrvatske, predstavlja logičan korak u razvoju projekta.
		
		Iskustvo rada na ovom projektu bilo je izuzetno korisno za sve članove tima. Poboljšali smo ne samo tehničke vještine već i sposobnosti komunikacije, planiranja i timskog rada. Sudjelovanje u projektu pružilo nam je uvid u dinamiku rada u timu te nas pripremilo za buduće izazove u našim karijerama. Zadovoljni smo rezultatom ovog projekta i što smo bili njegov dio te se veselimo nastavku učenja i rasta u budućnosti.
		
		\eject 